\documentclass[mathserif]{beamer}
\usetheme{metropolis}

\usepackage[utf8]{inputenc}
\usecolortheme{lily} % Beamer color theme
\usepackage[english]{babel}
\usepackage{kotex}
\usepackage{amsmath,amsfonts,amssymb}
\usepackage{verbatim}


\title{코로나 전/후의 대여소별 자전거 대여량에 대한 베이지안 다중 검정}
\author{이동균}
\date{\today}
% \institute{University of Seoul}

\begin{document}

\begin{frame}
    \titlepage
\end{frame}

\begin{frame}{목차}
\begin{enumerate}
    \item 문제 설명
    \item 데이터 전처리
    \item EDA
    \item 분석방법 및 결과
\end{enumerate}
\end{frame}

\begin{frame}{1. 문제 설명}
\begin{itemize}
    \item 2020년 2월 이후의 집단적인 코로나 감염과 3월 초의 위기경보 격상을 통해서 코로나에 대한 위기감을 인지한 사람들의 외출 자제로 인한 자전거 대여량의 차이를 대여소별로 베이지안 다중검정을 실시하여 차이를 파악해본다.
    \vspace{3mm}
    \item 코로나 바이러스 발생 이후 대여소별로 자전거 대여량이 얼마나 차이가 있는지를 알아볼 수 있다. (여기서는 2019년 $3\sim5$월의 대여량과 2020년 $3\sim5$월의 대여량을 비교하였다.)
\end{itemize}
\end{frame}

\begin{frame}{1. 문제 설명}
- 검정하고자 하는 문제 : 대여소 별로 일별 대여량의 차이의 평균에 대한 사후 분포를 구하고, 검정   
\begin{itemize}
    \item 사전분포 : 1421개 대여소 별 대여량 차이의 평균을 구하고, 그 평균들의 분포를 사전분포로 정의
    \item 관측치 : 대여소 별로 일별 대여량의 차이(2020년 해당 날짜의 대여량 - 2019년 해당 날짜의 대여량) 관측치가 된다.
\end{itemize}
\end{frame}


\begin{frame}{2. 데이터 전처리}
\begin{enumerate}
    \item 서울시 열린 데이터 사이트의 서울시 공공 자전거 서울시 공공자전거 이용현황에서 대여이력정보 데이터를 수집
    \item 대여소별로 일일 대여량의 차이(2020년 날짜의 대여량- 2019년 날짜의 대여량)를 Column 값으로 가지도록 R 프로그램을 이용하여 전처리
\end{enumerate}
\begin{table}[]
\resizebox{\textwidth}{!}{%
\begin{tabular}{|c|c|c|c|c|c|c|c|}
\hline
Index                 & 관측소 번호                & 위도                    & 경도                    & 03/01                 & 03/02                 & \cdots & 05/31                 \\ \hline
1                     & 301                   & 37.57579              & 126.9715              & -1                    & -27                   & \cdots & -33                   \\ \hline
2                     & 302                   & 37.57595              & 126.9741              & -20                   & -34                   & \cdots & -102                  \\ \hline
\vdots & \vdots & \vdots & \vdots & \vdots & \vdots & \vdots & \vdots \\ \hline
1421                  & 1062                  & 37.55108              & 127.1626              & 6                     & 0                     & \cdots & -9                    \\ \hline
\end{tabular}%
}
\end{table}
\end{frame}

\begin{frame}{3. EDA}
- 서울시 일별 평균 대여량 차이의 히스토그램 및 QQplot
\begin{figure}%
\subfloat{{\includegraphics[width=0.45\textwidth ]{2.PNG} }}%
\subfloat{{\includegraphics[width=0.45\textwidth ]{3.PNG} }}%
\end{figure}
\end{frame}

\begin{frame}{3. EDA}
- 대여소별 일별 대여량 차이의 히스토그램
\begin{figure}%
\subfloat{{\includegraphics[width=0.45\textwidth ]{4.PNG} }}%
\subfloat{{\includegraphics[width=0.45\textwidth ]{5.PNG} }}%
\end{figure}
\end{frame}


\begin{frame}{4. 분석 방법 및 결과}
각 대여소별로 관측치와 사전분포는 정규분포이고 관측치의 분산($\sigma^2$)을 안다는 가정 하에 사후분포를 구해서 검정을 실시한다.
\begin{align*}
    \overline{X}|\theta & \sim N(\theta,\sigma^2/n)\\
    \theta & \sim N(\mu_0,\sigma^2_0)\\
    \theta|\bar{x} & \sim N\left(\frac{\frac{1}{\sigma^2\slash n}\bar{x}+\frac{1}{\sigma^2_0}\mu_0}{\frac{1}{\sigma^2\slash n}+\frac{1}{\sigma^2_0}},\left(\frac{1}{\sigma^2\slash n}+\frac{1}{\sigma^2_0}\right)^{-1}\right)
\end{align*}

\begin{itemize}
    \item $\theta$ 사전분포의 평균과 분산은 서울시의 대여소별 일별대여량 차이의 평균값들의 평균과 분산으로 가정하였다.
    \item 대여소별로 분산을 추정하여 $\sigma^2$ 에 해당하는 값으로 가정하였다.
\end{itemize}
\end{frame}

\begin{frame}{4. 분석 방법 및 결과}
- Parameter 값 구하기\\
\vspace{3mm}
사전분포의 평균($\mu_0$) : $-12.63978$\\ 분산($\sigma^2_0$) : $154.2494$\\ $n : 92$
\begin{table}[]
\resizebox{\textwidth}{!}{%
\begin{tabular}{|c|c|c|c|c|c|c|c|c}
\cline{1-8}
Index                 & 관측소 번호                & 위도                    & 경도                    & 관측치 평균                & 관측치 분산                & 사후분포 평균               & 사후분포 분산               &  \\ \cline{1-8}
1                     & 301                   & 37.57579              & 126.9715              & -19.4891304           & 449.17570             & -18.8605004           & 4.7325487             &  \\ \cline{1-8}
2                     & 302                   & 37.57595              & 126.9741              & -34.3260870           & 1172.04634            & -22.4172712           & 6.5546430             &  \\ \cline{1-8}
\vdots & \vdots & \vdots & \vdots & \vdots & \vdots & \vdots & \vdots &  \\ \cline{1-8}
1421                  & 1062                  & 37.55108              & 127.1626              & -0.3478261            & 27.96560              & -0.3451752            & 0.3033761             &  \\ \cline{1-8}
\end{tabular}%
}
\end{table}
\end{frame}

\begin{frame}{4. 분석방법 및 결과}
각 대여소별로 가설은 아래와 같이 설정하고, 각각의 사후확률을 구하여 가장 높게 추정되는 값에 해당하는 가설을 채택한다.
\begin{align*}
    H_1 & : \theta < -12.63978 \\
    H_2 & : -12.63978 \leq \theta < 0\\
    H_3 & : \theta \geq 0
\end{align*}
\end{frame}

\begin{frame}{4. 분석방법 및 결과}
- 각각 대여소별로 사후확률이 높은 가설을 하나씩 채택한다.
% Please add the following required packages to your document preamble:
% \usepackage{graphicx}
\begin{table}[]
\resizebox{\textwidth}{!}{%
\begin{tabular}{|c|c|c|c|c|c|c|c|}
\hline
Index                 & 관측소 번호                & 위도                    & 경도                    & p1                                                         & p2                                                        & p3                                                        & 채택                    \\ \hline
1                     & 301                   & 37.57579              & 126.9715              & 0.98863                                                    & 1.137037 \times 10^{-3} & 0                                                         & 1                     \\ \hline
2                     & 302                   & 37.57595              & 126.9741              & 1                                                          & 0                                                         & 0                                                         & 1                     \\ \hline
3                     & 303                   & 37.57177              & 126.9747              & 0.9999739                                                  & 2.611479 \times 10^{-5} & 0                                                         & 1                     \\ \hline
\vdots & \vdots & \vdots & \vdots & \vdots                                      & \vdots                                     & \vdots                                     & \vdots \\ \hline
1420                  & 1061                  & 37.54693              & 127.1346              & 3.742430 \times 10^{-6}  & 0.9999402                                                 & 5.610641 \times 10^{-5} & 2                     \\ \hline
1421                  & 1062                  & 37.55108              & 127.1626              & 3.38965 \times 10^{-110} & 0.752859                                                  & 0.2497141                                                 & 2                     \\ \hline
\end{tabular}%
}
\end{table}
\end{frame}

\begin{frame}{4. 분석방법 및 결과}
\scriptsize 대부분의 대여소가 첫번째 혹은 두번째 가설을 채택하므로 대부분의 대여소에서 대여량이 줄어들었다고 볼 수 있다. 그리고 영등포구, 마포구, 중구, 광진구 등의 지역에서는 서울시의 평균보다도 더 많이 대여량이 줄었다고 볼 수 있다.
    \begin{figure}
        \centering
        \includegraphics[width = 0.6\textwidth]{bayes_spring2.PNG}
        \caption{\sqriptsize 보라 : 1번째 가설 / 청록: 2번째 가설 / 노랑 : 3번째 가설}
    \end{figure}
\end{frame}




\end{document}